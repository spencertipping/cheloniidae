\documentclass{article}

\usepackage{palatino}
\usepackage{amsmath}
\usepackage{mathpazo}
\usepackage{listings}
\usepackage{hyperref}

\lstnewenvironment{java}
  {\lstset{language=java,basicstyle={\tt\scriptsize}}}
  {}

\newenvironment{commandtable}
  {\begin{enumerate}}
  {\end{enumerate}}

\def\command[#1]{\item[\tt #1]}
\def\example[#1]{\par Example: {\tt #1}}

\begin{document}
  \title{Terrapin Turtle Graphics}
  \author{Spencer Tipping}
  \date{}
  \maketitle

  \tableofcontents

  \section{Introduction}
    \label{sec:introduction}

    Terrapin turtle graphics is a capable, easy-to-use library for Java. It
    provides many advanced features such as three-dimensional space support
    under three different coordinate models, multiple turtles, antialiased and
    depth-adjusted rendering, highly customizable axes and grids, and
    perspective projection, all without losing the simplicity and elegance of
    pure vector graphics.

    This library originated as a drop-in replacement for the Galapagos turtle
    software, used primarily for educational purposes. As such, there are two
    sets of commands: One set uses the same nomenclature and functionality as
    the original Galapagos library, and the other uses my nomenclature and
    supports the extensions listed previously. For more details, see section
    \ref{sec:command-index}.

  \section{Coordinate Models}
    \label{sec:coordinate-models}

    Terrapin seamlessly integrates two-dimensional and three-dimensional
    drawing. By default, a turtle draws only on the plane and the model is for
    all purposes two-dimensional. However, by changing the turtle's $\phi$
    heading, the turtle begins to change its $z$ position.

    Depending on the drawing, different coordinate models will make sense. For
    example, drawing a sphere such as those in the included example {\tt
    spheres.java} is most easily achieved using spherical coordinates. However,
    spherical coordinates make it very difficult to draw a spiral. There are
    similar tasks that are well-suited to other models, so three separate
    coordinate models are provided. Others may be written for Terrapin without
    very much work, especially by inheriting from the {\tt Turtle} class.

    \subsection{Spherical Coordinates}
      \label{sec:z-spherical}
      \label{sec:spherical-coordinates}

      Spherical coordinates allow the turtle's default plane to be bent into a
      cone whose depth angle is determined by $\phi$. By default, $\phi=0$, so
      the plane is flat. Mathematically, the distance vector $\langle dx, dy,
      dz\rangle$ moved by the turtle for a move of distance $\langle d\rangle$
      is computed this way:

      \begin{align*}
	dx & = d \cos \theta \cos \phi \\
	dy & = d \sin \theta \cos \phi \\
	dz & = d \sin \phi
      \end{align*}

    \subsection{Cylindrical Coordinates}
      \label{sec:y-cylindrical}
      \label{sec:cylindrical-coordinates}

      Cylindrical coordinates allow the turtle's drawing plane to be rotated
      about the Y axis. The degree of rotation is determined by $\phi$, which is
      zero by default. Mathematically, cylindrical coordinates are implemented
      as follows (see section \ref{sec:z-spherical} for notation):

      \begin{align*}
	dx & = d \cos \theta \cos \phi \\
	dy & = d \sin \theta \\
	dz & = d \cos \theta \sin \phi
      \end{align*}

    \subsection{Orthogonal Planar Coordinates}
      \label{sec:orthogonal-planar}

  \section{Command Index}
    \label{sec:command-index}

    This section lists all of the commands supported by the Turtle class in
    Terrapin. All of the examples assume the following definitions:

    \lstset{gobble=6}
    \begin{java}
      import terrapin.*;

      public class test {
	public static void main (String[] args) {
	  TurtleDrawingWindow w = new TurtleDrawingWindow ();
	  Turtle t = new Turtle ();
	  w.add (t);
	  
	  // Any example would be legal here.

	  w.setVisible (true);
	}
      }
    \end{java}

    \subsection{Basic Commands}
      \label{sec:basic-commands}

      These commands allow the turtle to create basic shapes. If only these
      commands are used, then the turtle will remain in a two-dimensional plane;
      thus we postpone the three-dimensional details to section
      \ref{sec:3d-commands}.

      \begin{commandtable}
	\command[move]
	Moves the turtle a given distance along its heading, drawing a line if
	the pen is down. If the distance is negative, then the turtle moves
	backwards.
	\example[t.move(100);]

	\command[moveTo]
	Moves the turtle to a specific location, drawing a line if the pen is
	down. The turtle's heading is not taken into account and not changed by
	this operation.
	\example[t.moveTo(50, 10);]

	\command[jump]
	Moves the turtle a given distance along its heading without drawing a
	line. If the distance is negative, then the turtle moves backwards.
	\example[t.jump(100);]

	\command[jumpTo]
	Moves the turtle to a specific location without drawing a line. Like
	{\tt moveTo}, the turtle's heading is not considered for this operation.
	\example[t.jumpTo(10, 10);]

	\command[turn]
	Adds an angle to the turtle's heading. The turtle does not draw anything
	when it is turned. All headings are measured in degrees.
	\example[t.turn(90);]

	\command[setPenColor]
	Sets the color of future lines drawn by the turtle. The color may be
	translucent, in which case the drawn lines will also be translucent.
	\example[t.setPenColor(java.awt.Color.BLUE);]

	\command[setPenIsDown]
	Determines whether the turtle will draw lines when moved. By default,
	this is true.
	\example[t.setPenIsDown(true);]
      \end{commandtable}

    \subsection{3D Commands}
      \label{sec:3d-commands}

      These commands allow the turtle to travel anywhere in 3D space and change
      coordinate models.

      \begin{commandtable}
	\command[turnTheta]
	Rotates the turtle within its immediate plane or cone. The exact
	behavior of this command is determined by the coordinate model used.
	(See section \ref{sec:coordinate-models}.) This command is an alias for
	the {\tt turn} command. (See section \ref{sec:basic-commands}.)
	\example[t.turnTheta(-90);]

	\command[turnPhi]
	Adjusts the turtle's $\phi$ heading. The exact behavior of the $\phi$
	heading depends on the coordinate model. (See section
	\ref{sec:coordinate-models}.)
	\example[t.turnPhi(45);]

	\command[turnXi]
	Adjusts the turtle's $\xi$ heading. This is meaningful only if the
	turtle is using the orthogonal-planar coordinate model. (See sections
	\ref{sec:orthogonal-planar} and \ref{sec:coordinate-models}.)
	\example[t.turnXi(30);]

	\command[setPolarAxisModel]
	Determines the roles of the turtle's 3D heading coordinates, $\theta$,
	$\phi$, and $\xi$. See section \ref{sec:coordinate-models} for a
	mathematical description of these roles. Valid settings are
	\verb|Turtle.Z_SPHERICAL|, \verb|Turtle.Y_CYLINDRICAL|, and
	\verb|Turtle.ORTHOGONAL_PLANAR|.
	\example[t.setPolarAxisModel(Turtle.Z\_SPHERICAL);]
      \end{commandtable}

\end{document}
